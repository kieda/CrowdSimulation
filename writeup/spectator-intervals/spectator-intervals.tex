\documentclass[11pt]{article}
\usepackage{amsfonts}
\usepackage{amsmath}
\setlength\parindent{0pt}
\begin{document}
\title{Multiple Spectator Problem}
\author{Zachary Kieda (zkieda@andrew.cmu.edu)}
\date{\today}
\maketitle

\begin{abstract}
\setlength\parindent{0pt}
\noindent 
We present an algorithm that coalesces a sorted sequence of intervals to find a holistic interval which does not include overlaps. This is doable in $O(n)$, which is an improvement from the $O(n\log n)$ scan-line algorithm. 
\textbf{Keywords.} Algorithms, Data Structures
\end{abstract}
\begin{section}{Problem Statement}
Consider a list of intervals $L$ that is ordered from front to back. Intervals may or may not overlap, but when they do an interval from the front takes precedence. For all intervals $i, j \in L$ where $i < j$ will have length $len(i) \le len(j)$. We are also provided with a left to right sorting $S$ on the intervals, such that for all $i,j\in L$, $S(i) < S(j)$ iff the lowest point on the interval $i$ is less than the lowest point on interval $j$. $S(k)$ represents the index of interval $k$.
\end{section}
\begin{section}{Proposed Solution}
We scan the list $L$ from front to back. We store an array $A$ that maps an index to a range. 
\begin{itemize}
\item For $l\in L$, Let $rng$ be a new interval with properties $hi$ and $lo$. 
\begin{itemize}
\item If $l$ is not the first index and the previous item in $A$ is non null, $rng.lo = max(A(S(l.prev).maxVal, l.lo)$. Otherwise, $rng.lo = l.lo$
\item If $l$ is not the last index and the next item in $A$ is non null, $rng.hi = min(A(S(l.next).minVal, l.hi)$. Otherwise, $rng.hi = l.hi$.
\item Set $A[l] = rng$
\end{itemize}
\item Scan through each $rng \in L$, if $rng.hi \le rng.lo$, remove the interval. (Or, do not append the item to a newly created array)
\item return $A$.
\end{itemize}
Where $l.maxVal := max(l.hi, l.lo)$, $l.minVal := min(l.hi, lo)$.\\

\textbf{Correctness.} Suppose we're on some $l\in L$. If the previous item in $A$ does not exist, then there is no element that is above 
this element and to the left. So, $rng.lo$ will be seen in our final interval range

%     X X X X 
% X X X X 
% -> 
%     X X X X
% X X X X X X 

% so

%     X X X X
%                 X X X X
%           X X X X
% -> 
%     X X X X
%                 X X X X 
%             X X 

%   X X X X
% X X X X 		<- this interval cannot possibly exceed that of its parent
%     X X X X 

%   X X X X X X     [0, 5]
%       X X X       [2, 4]
%             X X X [5, 7]

%   X X X X X X     [0, 5]
%       X X X X     [5, 4]
%             X X X [5, 7]

%   X X X X X X     [0, 5]
%       X X X X     [5, 4]
%               X X [6, 7]

% alt
%         X X X     [3, 5]
%       X X X       [2, 4]
%             X X X [5, 7]

%         X X X     [3, 5]
%       X X X X     [2, 4]
%             X X X [5, 7]

%         X X X     [3, 5]
%       X X X X     [2, 4]
%               X X [5, 7]

\end{section}
\end{document}